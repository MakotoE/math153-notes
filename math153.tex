\documentclass{article}
\usepackage{amsmath}
\usepackage{amsfonts}
\usepackage{parskip}
\usepackage{svg}
\usepackage[utf8]{inputenc}
\usepackage{helvet}
\renewcommand{\familydefault}{\sfdefault}
\usepackage{geometry}
\usepackage[document]{ragged2e}
\geometry{letterpaper, portrait, top=1in, bottom=1in, left=1.5in, right=1.5in}
\usepackage{comment}

\title{MATH 153}

\begin{document}
\section{Sequences}

A sequence is a list of numbers written in a defined order

\begin{gather*}
    a_1, a_2, a_3, \cdots
\end{gather*}

Sequences can also be denoted
\begin{gather*}
    \{a_n\} \\
    \{a_n\}^\infty_{n=1}
\end{gather*}

Converging sequences settle on a certain number while diverging sequences does not settle on a number.

Recursive sequence: A sequence that depends on previous values in the sequence.

\section{Limit of a sequence}

L'Hospital's rule: For differentiable functions $f$ and $g$, if $\lim_{x \to c} f(x) = \lim_{x \to c} g(x) = 0$ or $\pm \infty$, and $g'(x) \neq 0$ for all $x$, and $\lim_{x \to c} \frac{f'(x)}{g'(x)}$ exists, then

\begin{gather*}
    \lim_{x \to c} \frac{f(x)}{g(x)} = \lim_{x \to c} \frac{f'(x)}{g'(x)}
\end{gather*}

Limit laws:

Let $\{a_n\}$ and $\{b_n\}$ be convergent sequences and $c$ be a constant.

\begin{gather*}
\lim_{n \to \infty} (a_n + b_n) = \lim_{n \to \infty} a_n + \lim_{n \to \infty} b_n \\
\lim_{n \to \infty} (a_n - b_n) = \lim_{n \to \infty} a_n - \lim_{n \to \infty} b_n \\
\lim_{n \to \infty} (a_n b_n) = \lim_{n \to \infty} a_n \cdot \lim_{n \to \infty} b_n \\
\lim_{n \to \infty} ca_n = c \lim_{n \to \infty} a_n \\
\lim_{n \to \infty} \frac{a_n}{b_n} = \frac{\lim_{n \to \infty} a_n}{\lim_{n \to \infty} b_n} \text{ if } \lim_{n \to \infty} b_n \neq 0 \\
\lim_{n \to \infty} a^p_n = \left( \lim_{n \to \infty} a_n \right)^p \text{ if } p > 0, a_n > 0
\end{gather*}

If $\lim_{n \to \infty} |a_n| = 0$, $\lim_{n \to \infty} a_n = 0$.

If $\lim_{n \to \infty} a_n = L$ and function $f$ is continuous at $L$, $\lim_{n \to \infty} f(a_n) = f(L)$.

The sequence $\{r^n\}$ is convergent if $-1 < r \leq 1$ and divergent for all other values of $r$.

\begin{gather*}
    \lim_{n \to \infty} r^n = \begin{cases}
        0 \text{ if } -1 < r < 1 \\
        1 \text{ if } r = 1
    \end{cases}
\end{gather*}

Squeeze theorem: If $a_n \leq b_n \leq c_n$ and $\lim_{n \to \infty} a_n = \lim_{n \to \infty} c_n = L$, $\lim_{n \to \infty} b_n = L$.

Monotonic sequences: A sequence ${a_n}$ is increasing if $a_n < a_{n+1}$ for all $n \geq 1$ and is decreasing if $a_n > a_{n+1}$ for all $n \geq 1$. Monotonic sequences are either increasing or decreasing.

Bounded sequences: A sequence ${a_n}$ is bounded above if there is a number $M$ such that $a \leq M$ for all $n \geq 1$. ${a_n}$ is bounded below if there is a number $m$ such that $m \leq a_n$ for all $n \geq 1$. If ${a_n}$ is bounded above and below, then ${a_n}$ is a bounded sequence.

If a sequence is bounded and monotonic, that sequence is convergent.

\section{Ways to determine the limit}

\begin{itemize}
    \item Draw graph or construct table
    \item Break it up into smaller limits using the limit laws
    \item L'Hospital's Rule
    \item Determine how each term will change as $n$ approaches infinity
    \item Use the squeeze theorem if it converges between two functions with the same limits
    \item Substitute a variable if a common pattern appears more than once. Remember to solve its limit to replace the limit variable.
    \item If it is raised to the $n$th power, calculate the natural log of both sides.
\end{itemize}

\section{Series}

Series: The sum of all terms of a sequence.

Infinite sum:
\begin{gather*}
    \sum_{n=1}^{\infty} a_n = S
\end{gather*}

Partial sum:
\begin{gather*}
    \sum_{K=1}^{n} a_K = S_n
\end{gather*}

Partial sums can be used to determine convergence of an infinite sum.

\begin{gather*}
    \sum_{n=1}^{\infty} a_n = \lim_{n \to \infty} \sum_{K=1}^{n} a_K
\end{gather*}

If $\{S_n\}$ converges, then $\lim_{n \to \infty} S_n = S$.

\section{Geometric series}

\begin{gather*}
    \sum_{n=1}^{\infty} ar^{n-1} = a + ar + a^2 + a^3 + \cdots + ar^{n-1}
\end{gather*}

\begin{itemize}
    \item If $r \leq -1$ or $r \geq 1$, the series will diverge  \\
    \item If $-1 < r < 1$, the series will converge and the limits is \\
    \begin{gather*}
        S = \frac{a}{1-r}
    \end{gather*}
\end{itemize}

Theorems for convergent series
\begin{align*}
    \sum_{n=1}^{\infty} ca_n &= c \sum_{n=1}^{\infty} a_n \\
    \sum_{n=1}^{\infty} (a_n + b_n) &= \sum_{n=1}^{\infty} a_n + \sum_{n=1}^{\infty} b_n \\
    \sum_{n=1}^{\infty} (a_n - b_n) &= \sum_{n=1}^{\infty} a_n - \sum_{n=1}^{\infty} b_n
\end{align*}

Telescoping series: Each pair of consecutive terms have parts that cancel out, which makes it easy to find the closed form.

Test for divergence: If $\lim_{n \to \infty} a_n$ does not exist or if $\lim_{n \to \infty} a_n \neq 0$, then the series $\sum_{n=1}^{\infty} a_n$ is divergent.

\section{Integral test}

Integral Test: Let $f$ be a continuous, positive, decreasing function on $[1, \infty)$ and let $a_n = f(n)$. Then the series $\sum_{n=1}^\infty a_n$ is convergent if and only if the improper integral $\int_1^\infty f(x) dx$ is convergent.

P-series test: The $p$-series $\sum_{n=1}^\infty \frac{1}{n^p}$ is convergent if $p > 1$ and divergent if $p \leq 1$.

\section{Comparison test}

Comparison test: Let $\sum a_n$ and $\sum b_n$ be series with positive terms.
\begin{itemize}
    \item If $\sum b_n$ is convergent and $a_n \leq b_n$ for all $n$, then $\sum a_n$ is convergent.
    \item If $\sum b_n$ is divergent and $a_n \geq b_n$ for all $n$, then $\sum a_n$ is divergent.
\end{itemize}

\section{Limit comparison test}

Limit comparison test: Let $\sum a_n$ and $\sum b_n$ be series with positive terms. If
\begin{gather*}
    \lim_{n \to \infty} \frac{a_n}{b_n} = c
\end{gather*}

where $c$ is a finite number and $c > 0$, then either both series converge or both diverge.

\section{Alternating series}

Alternating series: A series whose terms are alternately positive and negative.
\begin{gather*}
    \sum_{n = 0}^\infty (-1)^{n-1} b_n = a_0 - a_1 + a_2 - a_3 + \cdots
\end{gather*}

Alternating series test: Given an alternating series $\sum (-1)^{n-1} b_n$, if $b_n$ is decreasing and $\lim_{n \to \infty} b_n = 0$, then $\sum (-1)^{n-1} b_n$ is convergent.

Alternating series estimation theorem: If $S = \sum (-1)^{n-1}b_n$ where $b_n > 0$ is the sum of a converging alternating series, then
\begin{gather*}
    |R_n| = |S - S_n| \leq b_{n + 1}
\end{gather*}

\section{Ratio test}

Absolute convergence: Given a series $\sum a_n$, if $\sum |a_n|$ converges, then $\sum a_n$ is absolutely convergent.

Conditional convergence: A series $\sum a_n$ that is convergent but not absolutely convergent.

Ratio test: Given the series $\sum_{n=1}^\infty a_n$,
\begin{gather*}
    \text{if } \lim_{n \to \infty} \left| \frac{a_{n+1}}{a_n} \right|
    \begin{cases}
        < 1 \text{, then the series is absolutely convergent} \\
        > 1 \text{ or } \infty \text{, then the series is divergent} \\
        = 1 \text{, the Ratio Test is inconclusive}
    \end{cases}
\end{gather*}

\section{Root test}

Root test: Given the series $\sum_{n=1}^\infty a_n$,
\begin{gather*}
    \text{if } \lim_{n \to \infty} |a_n|^{\frac{1}{n}}
    \begin{cases}
        < 1 \text{, then the series is absolutely convergent} \\
        > 1 \text{ or } \infty \text{, then the series is divergent} \\
        = 1 \text{, the Ratio Test is inconclusive}
    \end{cases}
\end{gather*}

\section{Tests for convergence and divergence}

\begin{itemize}
    \item Geometric series
    \item Telescoping series
    \item Test for divergence
    \item Integral test
    \item P-series test
    \item Comparison test
    \item Limit comparison test
    \item Alternating series test
    \item Ratio test
    \item Root test
\end{itemize}

\section{Power series}

The power series:
\begin{gather*}
    \sum_{n=0}^\infty c_n (x - a)^n
\end{gather*}

This power series is centered around $a$.

For a given power series $\sum_{n=0}^\infty c_n (x - a)^n$, there are only three possibilities:
\begin{itemize}
    \item The series converges only when $x=a$
    \item The series converges for all x
    \item There is a positive number $R$ such that the series converges if $|x - a| < R$ and diverges if $|x - a| > R$
\end{itemize}

The interval of convergence is the interval for $x$ such that the series converges.

The radius of convergence is half the distance of the interval of convergence.

Sometimes, such as for the ratio test and root test, the endpoints must be checked for convergence.

\section{Representations of functions as power series}

\begin{gather*}
    \frac{1}{1-x} = \sum_{n=0}^\infty x^n \\
    \text{where } |x| < 1
\end{gather*}

Term-by-term differentiation and integration:

If the power series $\sum_{n=0}^\infty c_n (x - a)^n$ has radius of convergence $R > 0$, then the function $f$ defined by

\begin{gather*}
    f(x) = c_0 + c_1(x+a) + c_2(x-a)^2 + \cdots = \sum_{n=0}^\infty c_n (x - a)^n
\end{gather*}

is differentiable on the interval $(a - R, a + R)$ and

\begin{gather*}
    f'(x) = c_1 + 2 c_2 (x - a) + 3 c_3 (x - a)^2 + \cdots = \sum_{n=1}^\infty n c_n (x - a)^{n-1} \\
    \int f(x) dx = C + c_0 (x - a) + c_1 \frac{(x-a)^2}{2} + c_2 \frac{(x-a)^3}{3} + \cdots = C + \sum_{n=0}^\infty c_n \frac{(x - a)^{n+1}}{n + 1}
\end{gather*}

The radii of convergence of both equations are $R$.

\section{Taylor and Maclaurin Series}

If the power series representation at $a$ is
\begin{gather*}
    f(x) = \sum_{n=0}^\infty c_n (x - a)^n
\end{gather*}

then its Taylor series is 

\begin{gather*}
    f(x) = \sum_{n=0}^\infty \frac{f^{n}(a)}{n!} (x - a)^n
\end{gather*}

Maclaurin series: The Taylor series but with $a = 0$

Important Maclaurin series
\begin{align*}
    \frac{1}{1-x} &= \sum_{n=0}^\infty x^n, R = 1 \\
    e^x &= \sum_{n=0}^\infty \frac{x^n}{n!}, R = \infty \\
    \sin(x) &= \sum_{n=0}^\infty (-1)^n \frac{x^{2n+1}}{(2n+1)!}, R = \infty \\
    \cos(x) &= \sum_{n=0}^\infty (-1)^n \frac{x^{2n}}{(2n)!}, R = \infty \\
    \tan^{-1}(x) &= \sum_{n=0}^\infty (-1)^n \frac{x^{2n+1}}{2n+1}, R = 1 \\
    \ln(1 + x) &= \sum_{n=1}^\infty (-1)^{n-1} \frac{x^n}{n}, R = 1 \\
    (1 + x)^k &= 1 + kx + \frac{k(k-1)}{2!}x^2 + \frac{k(k-1)(k-2)}{3!}x^3 + \cdots = \sum_{n=0}^\infty {k \choose n} x^n, R = 1
\end{align*}

\section{Parametric equations}
Unlike rectangular equations, parametric equations have a third variable $t$.
It has an initial point and terminal point for the minimum and maximum $t$ values respectively.

First derivative (Slope):
\begin{gather*}
    \frac{dy}{dx} = \frac{\frac{dy}{dt}}{\frac{dx}{dt}} \\
    \text{if }\frac{dx}{dt} \neq 0
\end{gather*}

Second derivative (Concavity):
\begin{gather*}
    \frac{d^2 y}{dx^2} = \frac{\frac{d}{dt}\left( \frac{dy}{dx} \right)}{\frac{dx}{dt}}
\end{gather*}

If a curve is defined by $x = f(t), y = g(t), \alpha \leq t \leq \beta$, the area under the curve is
\begin{gather*}
    \int_\alpha^\beta g(t) f'(t) dt
\end{gather*}

The arc length is
\begin{gather*}
    \int_\alpha^\beta \sqrt{\left(\frac{dx}{dt}\right)^2 + \left(\frac{dy}{dt}\right)^2} dt
\end{gather*}

\section{Polar coordinates}

Converting from polar coordinates to Cartesian coordinates
\begin{gather*}
    x = r \cos(\theta) \\
    y = r \sin(\theta)
\end{gather*}

Converting from Cartesian to polar coordinates
\begin{gather*}
    r^2 = x^2 + y^2 \\
    \tan(\theta) = \frac{y}{x}
\end{gather*}

Distance formula
\begin{gather*}
    d = \sqrt{(x_2 - x_1)^2 + (y_2 - y_1)^2}
\end{gather*}

\begin{itemize}
    \item If $f(\theta) = f(-\theta)$, the curve is symmetric about the polar axis. \\
    \item If $f(\theta) = f(\theta + \pi)$, the curve is symmetric about the pole. \\
    \item If $f(\theta) = f(\pi - \theta)$, the curve is symmetric about $\theta = \pi / 2$.
\end{itemize}

Area of $R$:

\begin{gather*}
    \int_a^b \frac{1}{2}(f(\theta))^2 d\theta
\end{gather*}

Trig identities

\begin{align*}
    \sin^2(\theta) &= \frac{1-\cos(2\theta)}{2} \\
    \cos^2(\theta) &= \frac{1+\cos(2\theta)}{2}
\end{align*}

\section{Vectors}

A vector:
\begin{itemize}
    \item Has a magnitude and direction.
    \item Has an initial point and terminal point
    \item A vector from $A$ to $B$ is written $\overrightarrow{AB}$
\end{itemize}

Scalar multiplication of vectors: If $c$ is a scalar and $v$ is a vector, the scalar multiple $cv$ is the vector whose length is $|c|$ times the length of $v$ and whose direction is the same as $v$ if $c > 0$ and is opposite to $v$ if $c<0$. If $c=0$, then $cv=0$.

Vector addition: The sum of vectors $u$ and $v$, $u+v$, is a vector from the initial point of $u$ to the terminal point of $v$. $u + v = v + u$.

Position vector: A vector that represents a coordinate relative to an origin.

Given the points $A(x_1, y_1, z_1)$ and $B(x_2, y_2, z_2)$, the vector $\overrightarrow{AB} = \left\langle x_2 - x_1, y_2 - y_1, z_2 - z_1 \right\rangle$

Unit vector: Vectors that have a magnitude of $1$.

\begin{align*}
    \hat{i} &= \text{unit vector in $x$ direction} \\
    \hat{j} &= \text{unit vector in $y$ direction} \\
    \hat{k} &= \text{unit vector in $z$ direction}
\end{align*}

Magnitude of a vector:
\begin{gather*}
    |a| = \sqrt{a_1^2 + a_2^2}
\end{gather*}

\begin{gather*}
    \hat{u} = \frac{\overrightarrow{a}}{|\overrightarrow{a}|}
\end{gather*}

Equation of a sphere:
\begin{gather*}
    (x-h)^2 + (y-k)^2 + (z-l)^2 = r^2
\end{gather*}

\end{document}
